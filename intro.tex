\documentclass[10pt]{article}
\title{Notes from \emph{Operating Systems Three Easy Pieces}}
\author{Sachidananda Urs}
\begin{document}
\maketitle

\begin{abstract}
This document contains the introductory notes from the book OS Three
Easy Pieces. The book aims to teach operating systems by presenting two types of
projects ``system programming'' projects and ``XV6'' based projects. See
https://github.com/remzi-arpacidusseau/ostep-projects for more details.
\end{abstract}

\subsection*{Introduction}
The three pieces the book aims to teach are \emph{virtualization},
\emph{concurrency}, and \emph{persistence}.

When we talk about \emph{virtualization}, we look at the virtualization of
\textbf{CPU} and \textbf{memory}.

When we talk about \emph{concurrency}, we look at threads, locks, condition
variables, semaphores, etc. And common problems in concurrency, event based
concurrency, ...

When we talk about \emph{persistence} we talk about I/O devices, hard disk
drives, RAID, files and directories, file system implementation ...

We also look into the basics of distributed systems, NFS, and AFS.

\subsection*{Virtualization}
We try to understand the virtualization of \emph{CPU} and \emph{Memory}

\subsubsection*{CPU Virtualization}
We study the abstraction (a process), APIs, states of a process, the data
structures involved, limited direct execution, and scheduling. Scheduling is
studied extensively.

\subsubsection*{Memory Virtualization}
We study the abstraction of memory (address space), APIs, address translation,
segmentation, free space management, paging and TLBs, swap space, policies, case
studies (VAX/VMS, Linux, illumos).

\subsection*{Concurrency}
We study the importance of concurrency and why we even bother about it. And
emphasize on threads, thread APIs, locks, condition variables, semaphores, ...

\subsection*{Persistence}
We study the I/O devices, viz hard disk drives, SSDs, and RAIDs. And the
related software concepts of files and directories, file system implementation
case studies, FFS, data integrity, so on.

\newpage
\section*{Introduction to Operating Systems}
We try to answer a simple question \emph{``How does the operating system
  virtualize resources?''}

Breaking the question even further we ask:
\begin{enumerate}
  \item What mechanism and policies are implemented by the operating system to
    attain virtualization?
  \item How does the operating system do it efficiently?
  \item What hardware support is needed?
\end{enumerate}

\subsection*{On Virtualization}
Virtualization is a technique where operating system takes a physical resource
(processor, memory, or disk) and transforms it into a more general, powerful,
and easy-to-use virtual form of itself.

Thus operating system is sometimes referred to as a virtual machine.

Now the question arises, how to use these resources? Or how to interact with the
operating system? This is done through a set of APIs called ``System calls''.

Summarily, virtualization allows many programs to run (sharing CPU), many
programs can concurrently access their data and instructions (sharing memory),
many programs can access devices (by sharing disks, network etc).

We examine some of the examples for:
\begin{enumerate}
  \item Virutalizing the CPU
  \item Virtualizing the memory
  \item Concurrency
  \item Persistence
\end{enumerate}

\subsubsection*{Virtualizing the CPU}
Consider a program which sleeps for a second and prints a string. Now, if the
program is run multiple times parallelly, it appears to work simultaneously
though we have a single CPU. How is this achieved? The OS provides an illusion
of multiple virtual CPUs. The operating system virtualizes the CPU.

Now, since the operating system runs multiple programs ``simultaneously'', if
two programs want to run at a particular time, which one should run first?

\subsubsection*{Virtualizing the memory}
We try to answer the question, \emph{``What is memory virtualizing?''}

Each process accesses its own private \emph{virtual address space}, which the
operating system somehow maps to physical memory of the machine. Memory
reference within one running program does not affect the address space of the
other process. As far as running program is concerned it has the entire memory
for itself. However, the physical memory is a shared resource, managed by the
operating system.

How does the operating system manage all this stuff is the concept called the
virtual memory.

\subsubsection*{Concurrency}
If a threaded program is written carelessly, concurrency leads to all sorts of
problems. The program does not behave as expected. By understanding concurrency
we can write these programs in a much better way.

\subsubsection*{Persistence}
When we talk about persistence we mean file systems which manage the data storage
and retrieval. We try to understand:
\begin{enumerate}
\item How is a file system written?
\item What techniques are needed to manage persistent data correctly?
\item How is high performance achieved?
\item How is reliability achieved?
\end{enumerate}
\end{document}
